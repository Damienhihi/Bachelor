\subsubsection{Grundlagen}
Der Begriff \glqq Künstliche Intelligenz\grqq{} wurde erstmals 1955 von John McCarthy, Marvin Minsky, Nathaniel Rochester und Claude Shannon im Rahmen der Vorbereitung auf die Dartmouth Conference geprägt. Darin beschrieben die Autoren Künstliche Intelligenz als die wissenschaftliche Disziplin, die davon ausgeht, dass sich menschliche Intelligenz so präzise beschreiben lässt, dass sie von Maschinen nachgebildet werden kann. Dieses Konzept bildete die Grundlage für die ein Jahr später abgehaltene Dartmouth Summer Research Conference on Artificial Intelligence im Jahr 1956, welche somit den beginn der Forschung im Bereich der \ac{KI} darstellt. \cite[\pagef 12]{mccarthy_proposal_2006}

Mehr als fünf Jahrzehnte später konkretisierte John McCarthy den Begriff \ac{KI} erneut. Er beschrieb sie als eine Wissenschaft und Technik der Herstellung intelligenter Maschinen und insbesondere intelligenter Computerprogramme. In seiner Ausführung wird deutlich, dass \ac{KI} als eigenständige Disziplin verstanden wird, die zwar in engem Zusammenhang mit dem Verständnis menschlicher Intelligenz steht, sich jedoch nicht ausschließlich an biologisch nachvollziehbaren Prozessen orientiert. \cite[\pagef 2]{mccarthy_what_2007}

Allerdings existieren, neben der Definition von McCarthy, auch viele weitere Definitionen von \ac{KI}, die je nach Forschungsrichtung und Anwendungsgebiet variieren. Eine allgemein akzeptierte und einheitliche Definition von \ac{KI} gibt es daher nicht. Nichtsdestotrotz enthalten die meisten Definitionen gemeinsame Bestandteile und somit ähnliche Hauptaussagen. \cite[\pagef 9]{goudz_grundlagen_2024}

Neben der 