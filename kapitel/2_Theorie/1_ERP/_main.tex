\subsection{ERP-Systeme}
%Einleitung in das Kapitel

Um das Grundlegende Verständnis von \ac{ERP}-System zu schaffen werden diese im Folgenden definiert, ihr Aufbau erläutert. Darauf aufbauend wird detailliert auf \ac{D365FnSCM} eingegangen, da dieses im methodischen Vorgehen als exemplarisches \ac{ERP}-System für die Implementierung des \ac{KI}-Agenten dient. Abschließend wird die historische Entwicklung von \ac{ERP}-Systemen dargestellt, um den Fortschritt zu modernen (KI-fähigen) Systemen zu verdeutlichen.

%Unterkapitel
\subsubsection{Definition und Zweck}
Ein \ac{ERP}-System ist wörtlich ins Deutsche übersetzt ein \glqq Unternehmensressourcenplanungssystem\grqq{}. Laut Klaus et al. \cite[\pagef 142]{klaus_what_2000} handelt es sich dabei um eine umfassende Softwarelösung, welche darauf abzielt alle Prozesse und Funktionen eines Unternehmens zu integrieren und in einer einheitlichen Informations- und It-Architektur abzubilden, um dadurch eine ganzheitliche Sicht auf das Unternehmen zu ermöglichen.
\subsubsection{Aufbau}
\input{kapitel/2_Theorie/1_ERP/2_Aufbau.tex}
\subsubsection{Microsoft Dynamics 365 for Finance and Supply Chain Management} 
\subsubsection{Microsoft Dynamics 365 for Finance and Supply Chain Management} 
\subsubsection{Historische Entwicklung}
\input{kapitel/2_Theorie/1_ERP/4_Historie.tex}